\usepackage[dvipsnames]{xcolor}
\usepackage{amsmath}   % 数学公式
\usepackage{amsthm}    % 定理环境
\usepackage{amssymb}   % 更多公式符号
\usepackage{graphicx}  
\usepackage{wrapfig}   % 插图
\usepackage{mathrsfs}  % 数学字体
\usepackage{physics}   % 物理符号
\usepackage{booktabs}
\usepackage{tabularx}
\usepackage{longtable} % 支持跨页表格
\usepackage{array}     % 增强表格功能
\usepackage{multirow}
\usepackage{tikz}
\usepackage{tikz-3dplot} % 引入 tikz-3dplot 宏包
\usepackage{subcaption} % 用于子图排版
\usetikzlibrary{arrows.meta, positioning, shapes.geometric}
\usepackage{enumitem}  % 列表
\usepackage{geometry}  % 页面调整
\usepackage{ragged2e}
\usepackage{unicode-math}
\usepackage[hidelinks,
colorlinks=true,
allcolors=blue,
pdfstartview=Fit,
breaklinks=true]{hyperref}

\graphicspath{ {flg/},{../flg/}, {config/}, {../config/} }  % 配置图形文件检索目录
\linespread{1.5} % 行高

% 页码设置
\geometry{top=25.4mm,bottom=25.4mm,left=20mm,right=20mm,headheight=2.17cm,headsep=4mm,footskip=12mm}

% 设置列表环境的上下间距
\setenumerate[1]{itemsep=5pt,partopsep=0pt,parsep=\parskip,topsep=5pt}
\setitemize[1]{itemsep=5pt,partopsep=0pt,parsep=\parskip,topsep=5pt}
\setdescription{itemsep=5pt,partopsep=0pt,parsep=\parskip,topsep=5pt}

% 定理环境
% ########## 定理环境 start ####################################
\theoremstyle{definition}
\newtheorem{defn}{\indent 定义}[section]

\newtheorem{add}{\indent 补充}[section]

\newtheorem{lemma}{\indent 引理}[section]    % 引理 定理 推论 准则 共用一个编号计数
\newtheorem{thm}[lemma]{\indent 定理}
\newtheorem{corollary}[lemma]{\indent 推论}
\newtheorem{criterion}[lemma]{\indent 准则}

\newtheorem{proposition}{\indent 命题}[section]
\newtheorem{example}{\indent \color{SeaGreen}{例}}[section] % 绿色文字的 例 ,不需要就去除\color{SeaGreen}{}
\newtheorem*{zhu}{\indent 注}

% 两种方式定义中文的 证明 和 解 的环境:
% 缺点:\qedhere 命令将会失效【技术有限,暂时无法解决】
\renewenvironment{proof}{\par\textbf{证明:}\;}{\qed\par}
\newenvironment{solution}{\par{\textbf{解:}}\;}{\qed\par}
\newenvironment{yzh}{\par{\textbf{已知:}}\;}{}
\newenvironment{tui}{\par{\textbf{推导:}}\;}{\qed\par}

% 缺点:\bf 是过时命令,可以用 textb f等替代,但编译会有关于字体的警告,不过不影响使用【技术有限,暂时无法解决】

% ######### 定理环境 end  #####################################

% ↓↓↓↓↓↓↓↓↓↓↓↓↓↓↓↓↓ 以下是自定义的命令  ↓↓↓↓↓↓↓↓↓↓↓↓↓↓↓↓

% 用于调整表格的高度  使用 \hline\xrowht{25pt}
\newcommand{\xrowht}[2][0]{\addstackgap[.5\dimexpr#2\relax]{\vphantom{#1}}}

% 表格环境内长内容换行
\newcommand{\tabincell}[2]{\begin{tabular}{@{}#1@{}}#2\end{tabular}}

% 使用\linespread{1.5} 之后 cases 环境的行高也会改变,重新定义一个 ca 环境可以自动控制 cases 环境行高
\newenvironment{ca}[1][1]{\linespread{#1} \selectfont \begin{cases}}{\end{cases}}
% 和上面一样
\newenvironment{vx}[1][1]{\linespread{#1} \selectfont \begin{vmatrix}}{\end{vmatrix}}

\def\d{\textup{d}} % 直立体 d 用于微分符号 dx
\def\R{\mathbb{R}} % 实数域
%\newcommand{\bs}[1]{\boldsymbol{#1}}    % 加粗,常用于向量
%\newcommand{\ora}[1]{\overrightarrow{#1}} % 向量

% 数学 平行 符号
\newcommand{\pll}{\kern 0.56em/\kern -0.8em /\kern 0.56em}
\newcommand{\pa}{\partial}
\newcommand{\vvec}{\overset{\rightharpoonup\!\!\!\! \rightharpoonup}}
% 中文平行 parallel
\newcommand*\zhparallel{%
  \mathrel{\text{\tikz[baseline,line cap=round] \draw (0em,-0.3ex) -- (.4em,1.7ex) (.2em,-0.3ex) -- (.6em,1.7ex);}}%
}
\newcommand{\X}{\mathtt{X}}
\newcommand{\mT}{\raisebox{0.1ex}{$\scriptstyle -T$}} % 调整高度为 0.1ex
\newcommand{\lmT}{\raisebox{-0.85ex}{$\scriptstyle -T$}} % 调整高度为 -0.85ex
\newcommand{\mone}{\raisebox{0.1ex}{$\scriptstyle -1$}} % 调整高度为 0.1ex
\newcommand{\lmone}{\raisebox{-0.85ex}{$\scriptstyle -1$}} % 调整高度为 -0.85ex
\newcommand{\vsup}[1]{\raisebox{-0.1ex}{$\scriptstyle #1$}}
\newcommand{\lsup}[1]{\raisebox{-0.85ex}{$\scriptstyle #1$}}
\newcommand{\mathminus}{\!\!-\!\!} % 数学环境连字符


