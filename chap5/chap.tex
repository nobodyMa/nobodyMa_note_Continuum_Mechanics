\ifx\allfiles\undefined
\documentclass[12pt, a4paper, oneside, UTF8]{ctexbook}  %  这一句是新增加的
\usepackage[dvipsnames]{xcolor}
\usepackage{amsmath}   % 数学公式
\usepackage{graphicx}
\usetikzlibrary{arrows, calc, decorations.pathmorphing}
\newcommand{\pa}{\partial}
\newcommand{\vvec}{\overset{\rightharpoonup\!\!\!\! \rightharpoonup}}
\newcommand{\X}{\mathtt{X}}
\newcommand{\mT}{\raisebox{0.1ex}{$\scriptstyle -T$}} % 调整高度为 0.1ex
\newcommand{\lmT}{\raisebox{-0.85ex}{$\scriptstyle -T$}} % 调整高度为 -0.85ex
\newcommand{\mone}{\raisebox{0.1ex}{$\scriptstyle -1$}} % 调整高度为 0.1ex
\newcommand{\lmone}{\raisebox{-0.85ex}{$\scriptstyle -1$}} % 调整高度为 -0.85ex
\newcommand{\mathminus}{\!\!-\!\!} % 数学环境连字符
\newcommand{\vsup}[1]{\raisebox{-0.1ex}{$\scriptstyle #1$}}
\newcommand{\lsup}[1]{\raisebox{-0.85ex}{$\scriptstyle #1$}}

\begin{document}
%\title{\Huge{\textbf{赵爹《连续介质力学》笔记}}}
\author{作者:无名氏马}
\date{\today}
\maketitle                   % 在单独的标题页上生成一个标题

\thispagestyle{empty}        % 前言页面不使用页码
\begin{center}
    \Huge\textbf{前言}
\end{center}

    本笔记根据
    \href{https://www.bilibili.com/video/BV1c54y1W78q/?spm_id_from=333.1387.upload.video_card.click&vd_source=0745441b4a83ceba73d32af3b7b0a955}{赵亚溥老师2020年春季《连续介质力学》课程}
    和教材
    (赵亚溥. 理性力学教程. 北京: 科学出版社, 2020.)整理而成,仅供参考学习。

\begin{flushright}
    \begin{tabular}{c}
        \today
    \end{tabular}
\end{flushright}

\newpage                      % 新的一页
\pagestyle{plain}             % 设置页眉和页脚的排版方式(plain:页眉是空的,页脚只包含一个居中的页码)
\setcounter{page}{1}          % 重新定义页码从第一页开始
\pagenumbering{Roman}         % 使用大写的罗马数字作为页码
\tableofcontents              % 生成目录

\newpage                      % 以下是正文
\pagestyle{plain}
\setcounter{page}{1}          % 使用阿拉伯数字作为页码
\pagenumbering{arabic}
% \setcounter{chapter}{-1}    % 设置 -1 可作为第零章绪论从第零章开始
 % 单独编译时,其实不用编译封面目录之类的,如需要不注释这句即可
\else
\fi
%  ↓↓↓↓↓↓↓↓↓↓↓↓↓↓↓↓↓↓↓↓↓↓↓↓↓↓↓↓ 正文部分
\chapter{构形configuration}
\section{class 31}
\begin{add}
数学上的流形manifold和物质流形material manifold的联系(\textbf{此部分内容AIGC})

\begin{center}
    \textbf{数学上的流形(Manifold)}
\end{center}
数学中的流形是一个\textbf{局部欧几里得空间化}的拓扑空间,其定义为:
\begin{itemize}
  \item \textbf{局部坐标映射}:对任意点 \( p \in M \),存在邻域 \( U \subseteq M \) 和同胚映射 \( \varphi: U \to \varphi(U) \subseteq \mathbb{R}^n \)。
  \item \textbf{光滑结构}:若坐标卡 \( (U_\alpha, \varphi_\alpha) \) 与 \( (U_\beta, \varphi_\beta) \) 的交集非空,则转移映射 \( \varphi_\beta \circ \varphi_\alpha^{-1} \) 需为 \( C^\infty \) 光滑。
  \item \textbf{全局性质}:流形可具有复杂拓扑(如紧致性、亏格),但不依赖嵌入更高维空间。
\end{itemize}
\noindent 典型例子包括球面 \( S^n \)、环面 \( T^n \),以及在物理中描述时空(广义相对论)或相空间(经典力学)。

\begin{center}
    \textbf{物质流形(Material Manifold)}
\end{center}
物质流形是连续介质力学中描述材料\textbf{未变形状态}的几何对象,其特性为:
\begin{itemize}
  \item \textbf{参考构形}:记为 \( \mathcal{B}_0 \),代表材料初始状态的3维流形,每个点 \( \mathbf{X} \in \mathcal{B}_0 \) 对应材料粒子。
  \item \textbf{物质坐标}:通过坐标映射 \( \mathbf{X} = (X^1, X^2, X^3) \) 唯一标记粒子。
  \item \textbf{变形映射}:随时间 \( t \),材料变形由光滑映射 \( \phi_t: \mathcal{B}_0 \to \mathcal{B}_t \subseteq \mathbb{R}^3 \) 描述,将物质坐标 \( \mathbf{X} \) 映射到空间坐标 \( \mathbf{x} = \phi(\mathbf{X}, t) \)。
\end{itemize}

\begin{center}
    \textbf{数学流形与物质流形的联系}
\end{center}
\textbf{几何结构}
物质流形 \( \mathcal{B}_0 \) 是3维微分流形,其坐标卡对应材料标签系统。数学流形的光滑结构允许定义:
\begin{itemize}
  \item \textbf{变形梯度张量}:\( F = \nabla_X \phi \),描述局部变形,满足 \( F_{iJ} = \frac{\partial x_i}{\partial X_J} \)。
  \item \textbf{质量守恒}:通过雅可比行列式 \( J = \det F \) 表达为 \( \rho_t J = \rho_0 \),其中 \( \rho_0 \) 为参考构形密度。
\end{itemize}

\textbf{映射的微分几何}
变形映射 \( \phi \) 需满足微分同胚(弹性力学)或允许不可逆性(塑性力学):
\begin{equation}
  \phi \in \text{Diff}(\mathcal{B}_0, \mathcal{B}_t) \quad \text{(弹性变形)}.
\end{equation}
塑性变形中,\( \phi \) 可能破坏微分同胚性,导致物质流形上出现\textbf{非完整结构}(如位错对应的非可积Burgers回路)。

\textbf{张量场与物理量}
\begin{itemize}
  \item \textbf{物质描述}:在 \( \mathcal{B}_0 \) 上定义Piola应力 \( P(\mathbf{X}, t) \),满足 \( P = J \sigma F^{-\top} \),其中 \( \sigma \) 为柯西应力。
  \item \textbf{空间描述}:在 \( \mathcal{B}_t \) 上定义速度场 \( \mathbf{v}(\mathbf{x}, t) \),通过李导数 \( \mathcal{L}_v \sigma \) 描述应力演化。
\end{itemize}

\textbf{拓扑与缺陷}
物质流形的拓扑不变量对应材料缺陷:
\begin{itemize}
  \item \textbf{位错}:Burgers矢量 \( \mathbf{b} \) 通过闭合回路 \( \gamma \subseteq \mathcal{B}_t \) 的不可闭合性定义:
  \[
  \mathbf{b} = \oint_\gamma d\mathbf{x} = \oint_{\phi^{-1}(\gamma)} F \, d\mathbf{X}.
  \]
  \item \textbf{裂纹}:物质流形的边界或非光滑结构对应宏观断裂。
\end{itemize}
\begin{example}
圆柱体的均匀膨胀与拉伸

考虑一个圆柱体材料(如橡胶圆柱),在受力后发生均匀的径向膨胀和轴向拉伸,使用 Material Manifold 描述其变形过程。

\textbf{1. 参考构形(未变形状态)}
\begin{itemize}
  \item 几何描述:
    \begin{itemize}
      \item 径向坐标:\( R \in [0, R_0] \),\( R_0 \) 为初始半径
      \item 轴向坐标:\( Z \in [0, L_0] \),\( L_0 \) 为初始长度
      \item 角度坐标:\( \Theta \in [0, 2\pi) \)
    \end{itemize}
  \item 坐标系:\( (R, \Theta, Z) \)
\end{itemize}

\textbf{2. 当前构形(变形状态)}
\begin{itemize}
  \item 变形后几何描述:
    \begin{itemize}
      \item 径向膨胀:\( r = \lambda_r R \)(\(\lambda_r\) 为径向拉伸比)
      \item 轴向拉伸:\( z = \lambda_z Z \)(\(\lambda_z\) 为轴向拉伸比)
      \item 角度不变:\( \theta = \Theta \)
    \end{itemize}
  \item 坐标系:\( (r, \theta, z) \)
\end{itemize}

\textbf{3. 变形梯度张量 \(\vvec{F}\)}
在柱坐标系中,变形梯度张量 \(\vvec{F}\) 的表达式为:
\[
\vvec{F} = \frac{\partial \mathbf{x}}{\partial \mathbf{X}} = 
\begin{bmatrix}
  \frac{\partial r}{\partial R}       & \frac{1}{R}\frac{\partial r}{\partial \Theta} & \frac{\partial r}{\partial Z} \\
  r\frac{\partial \theta}{\partial R} & \frac{r}{R}\frac{\partial \theta}{\partial \Theta} & r\frac{\partial \theta}{\partial Z} \\
  \frac{\partial z}{\partial R}       & \frac{1}{R}\frac{\partial z}{\partial \Theta} & \frac{\partial z}{\partial Z}
\end{bmatrix}
\]
对于均匀变形(无剪切和旋转):
\[
\vvec{F} = 
\begin{bmatrix}
  \lambda_r & 0         & 0 \\
  0         & \lambda_r & 0 \\
  0         & 0         & \lambda_z
\end{bmatrix}
\]
其中:
\begin{itemize}
  \item \(\lambda_r = \frac{r}{R}\) 为径向拉伸比
  \item \(\lambda_z = \frac{z}{Z}\) 为轴向拉伸比
\end{itemize}

\textbf{4. Material Manifold 的作用}
\begin{itemize}
  \item 参考构形 \((R, \Theta, Z)\) 描述未变形几何。
  \item 当前构形 \((r, \theta, z)\) 描述变形后几何。
  \item 通过 \(\vvec{F}\) 建立两构形间的映射,量化局部变形。
\end{itemize}

\textbf{5. 物理意义}
\begin{itemize}
  \item 若 \(\lambda_r = \lambda_z = 1\):无变形。
  \item 若 \(\lambda_r > 1, \lambda_z > 1\):径向膨胀与轴向拉伸并存。
  \item 若 \(\lambda_r \neq \lambda_z\):非均匀变形(如各向异性膨胀)。
\end{itemize}

\textbf{6. 变形前后圆柱体对比图}
\begin{center}
\begin{tikzpicture}[scale=1]
  % 未变形圆柱体
  \draw[thick, fill=blue!20] (0,0) ellipse (1 and 0.5);
  \draw[thick] (-1,0) -- (-1,2);
  \draw[thick] (1,0) -- (1,2);
  \draw[thick, fill=blue!20] (0,2) ellipse (1 and 0.5);
  \node at (0,-1.5) {参考构形};
  \draw[->, thick] (1.2,0.7) -- (2.2,0.7) 
  node[midway, above] {变} 
  node[midway, below] {形};

  % 变形后圆柱体
  \begin{scope}[xshift=4cm]
    \draw[thick, fill=red!20] (0,0) ellipse (1.5 and 0.75);
    \draw[thick] (-1.5,0) -- (-1.5,3);
    \draw[thick] (1.5,0) -- (1.5,3);
    \draw[thick, fill=red!20] (0,3) ellipse (1.5 and 0.75);
    \node at (0,-1.5) {当前构形};
  \end{scope}

  % 标注尺寸
  \draw[<->, thick] (-1.2,-0.7) -- (1.2,-0.7) node[midway, below] {\(2R_0\)};
  \draw[<->, thick] (-1.7,0) -- (-1.7,2) node[midway, left] {\(L_0\)};
  \draw[<->, thick] (4-1.7,0) -- (4-1.7,3) node[midway, left] {\(\lambda_z L_0\)};
  \draw[<->, thick] (4+1.7,-0.7) -- (4-1.7,-0.7) node[midway, below] {\(2\lambda_r R_0\)};
\end{tikzpicture}
\end{center}
\end{example}
\end{add}
% \begin{add}
%     mechanies on the material manifold

% 构形configuration,构形
% \end{add}
\begin{defn}
	变形函数$\vec{x}$
	\begin{gather*}
	\vec{x}=\vec{\mathcal{X}}(\vec{\X},t)\\
	inverse\;mapping\quad
	\vec{\X}=\vec{\mathcal{X}}^{-1}(\vec{x},t)
	\end{gather*}
	\begin{align*}
		d\vec{x}&=\vec{\mathcal{X}}(\vec{\X}+d\vec{\X},t)-\vec{\mathcal{X}}(\vec{\X},t)\\
		&=\frac{\pa \vec{\mathcal{X}}}{\pa \vec{\X}}\cdot d\vec{\X}
		=\vvec{F}d\vec{\X}=d\vec{\X}\vvec{F^T}\\
		\vec{x}=x_i\vec{e_i},\vec{\X}=\X_K\vec{e_K}\\
		\vvec{F}&=\frac{\pa \vec{x}}{\pa \vec{\X}}=\vec{x}\otimes\vec{\nabla}_{\vec{\X}}
		\quad\text{对参考构形的右梯度}\\
		&=\frac{\pa x_i}{\pa \X_K}\vec{e_i}\otimes\vec{e_K}
		\quad\text{两点张量场two\textminus point tensor}\\
		\vvec{F^T}&=\frac{\pa x_i}{\pa \X_K}\vec{e_K}\otimes\vec{e_i}\\
		inverse\;mapping\quad
		\vec{\X}=\vec{\mathcal{X}}^{-1}(\vec{x},t)\\
		d\vec{\X}&=\vec{\mathcal{X}}^{-1}(\vec{x}+d\vec{x},t)
		-\vec{\mathcal{X}}^{-1}(\vec{x},t)\\
		&=\vvec{F}^{\mone}d\vec{x}=d\vec{x}\vvec{F}^{\mT}\\
		\vvec{F}^{\mone}&=\frac{\pa \X_L}{\pa x_m}\vec{e_L}\otimes\vec{e_m}
	\end{align*}
	\begin{align*}
		\vvec{F}\vvec{F}^{\lmone}&=\left(\frac{\pa x_i}{\pa \X_K}\vec{e_i}\otimes\vec{e_K}\right)
		\cdot\left(\frac{\pa \X_L}{\pa x_m}\vec{e_L}\otimes\vec{e_m}\right)\\
		&=\frac{\pa x_i}{\pa \X_K}\frac{\pa \X_L}{\pa x_m}\delta_{KL}\vec{e_i}\otimes\vec{e_m}\\
		&=\frac{\pa x_i}{\pa x_m}\vec{e_i}\otimes\vec{e_m}
	\end{align*}
\end{defn}
\begin{defn}
	二阶张量的分类

	1.两个基矢量均在参考构形 Lagrangian type

	2.两个基矢量均在当前构形 Eulerian type

	3.两个基矢量在不同构形中
	$\begin{cases}
		mixed\; Lagrangian\mathminus Eulerian\; type\\
		mixed\; Eulerian\mathminus Lagrangian\; type
	\end{cases}$
\end{defn}
\begin{example}
	right Cauchy\textminus Green deformation tensor (拉格朗日型)
	\[\vvec{C}=\vvec{F^T}\vvec{F}=
	\left(\frac{\pa x_i}{\pa \X_K}\vec{e_K}\otimes\vec{e_i}\right)\cdot
	\left(\frac{\pa \vec{x_j}}{\pa \vec{\X_L}}\vec{e_j}\otimes\vec{e_L}\right)=
	\frac{\pa x_i}{\pa \X_K}\frac{\pa \vec{x_i}}{\pa \vec{\X_L}}\vec{e_K}\vec{e_L}\]
	\[\vvec{C^T}=(\vvec{F^T}\vvec{F})^T=\vvec{F^T}\vvec{F}=\vvec{C}\]
\end{example}
\begin{example}
	left Cauchy\textminus Green deformation tensor (欧拉型)
	\[\vvec{B}=\vvec{F}\vvec{F^T}=
	\left(\frac{\pa x_i}{\pa \X_K}\vec{e_i}\otimes\vec{e_K}\right)\cdot
	\left(\frac{\pa \vec{x_j}}{\pa \vec{\X_L}}\vec{e_L}\otimes\vec{e_j}\right)=
	\frac{\pa x_i}{\pa \X_K}\frac{\pa \vec{x_j}}{\pa \vec{\X_K}}\vec{e_i}\vec{e_j}\]
	\[\vvec{B^T}=(\vvec{F}\vvec{F^T})^T=\vvec{F}\vvec{F^T}=\vvec{B}\]
\end{example}
\begin{add}
	如何区分“左”和“右”?
\end{add}
\begin{example}
	Cauchy deformation tensor (欧拉型)
	\[\vvec{c}=\vvec{F}^{\lmT}\vvec{F}^{\lmone}=
	\left(\frac{\pa \X_K}{\pa x_i}\vec{e_i}\otimes\vec{e_K}\right)
	\cdot\left(\frac{\pa \X_L}{\pa x_j}\vec{e_L}\otimes\vec{e_j}\right)=
	\frac{\pa \X_K}{\pa x_i}\frac{\pa \X_K}{\pa x_j}\vec{e_i}\otimes\vec{e_j}\]
	\[\vvec{B}^{\lmone}=(\vvec{F}\vvec{F^T})^{-1}=\vvec{F}^{\lmT}\vvec{F}^{\lmone}=\vvec{c}\]
\end{example}
\begin{example}
	Finger tensor
\end{example}
\begin{defn}
	polar decomposition of deformation gradient
	\begin{gather*}
		\vvec{F}=\vvec{R}\vvec{U}=\vvec{V}\vvec{R}\\
		\vvec{R}\rightarrow orthogonal\quad\vvec{R^T}=\vvec{R}^{\lmone}\\
		\vvec{U}=\vvec{U^T},\vvec{v}=\vvec{v^T}\rightarrow symmetrical
	\end{gather*}

	right decomposition 
	\begin{gather*}
		\vvec{F^T}=(\vvec{R}\vvec{U})^T=\vvec{U}\vvec{R}^{\lmone}\\
		\sum_{\tau}\lambda_\tau^2\vec{e_\tau}\otimes\vec{e_\tau}
		=\vvec{C}=\vvec{F^T}\vvec{F}=\vvec{U}\vvec{R}^{\lmone}\cdot\vvec{R}\vvec{U}
		=\vvec{U}^2\\
		\vvec{U}=\sqrt{\vvec{C}}=\sum_{\tau}\lambda_\tau^2\vec{e_\tau}\otimes\vec{e_\tau}
	\end{gather*}

	left decomposition 
	\begin{gather*}
		\vvec{F}=\vvec{V}\vvec{R},\vvec{F}^{\lmT}=\vvec{R}^{\lmone}\vvec{V}\\
		\vvec{B}=\vvec{F}\vvec{F^T}=\vvec{V}\vvec{R}\cdot\vvec{R}^{\lmone}\vvec{V}=\vvec{V}^2\\
		\vvec{v}=\sqrt{\vvec{B}}=\sum_{\gamma}\lambda_\gamma\vec{e_\gamma}\otimes\vec{e_\gamma}
	\end{gather*}
\end{defn}
\section{class 32}
\begin{center}
	\textbf{四种常用应变的定义}
\end{center}
\begin{defn}
	engineering strain (small deformation)
\end{defn}
\begin{example}
	one\textminus dimensional

一维工程应变 \(\epsilon\) 定义为:
\[
\epsilon = \frac{\Delta L}{L_0} = \frac{L - L_0}{L_0}
\]
其中:
\begin{itemize}
    \item \( L_0 \) 是物体的原始长度,
    \item \( L \) 是物体变形后的长度,
    \item \( \Delta L = L - L_0 \) 是长度的变化量。
\end{itemize}
\end{example}
\begin{example}
	two\textminus dimensional(欧拉方法)

	在小变形假设下,二维工程应变的定义可以通过位移场的偏导数推导得出。假设物体在二维平面内发生变形,其位移场为 \( u(x, y) \) 和 \( v(x, y) \),分别表示在 \( x \) 和 \( y \) 方向上的位移。首先,定义位移梯度张量 \( \vvec{F} \) 为:
	\[
	\vvec{F} = \begin{bmatrix}
	\frac{\partial u}{\partial x} & \frac{\partial u}{\partial y} \\
	\frac{\partial v}{\partial x} & \frac{\partial v}{\partial y}
	\end{bmatrix}
	\]
	在小变形假设下,应变张量 \( \mathbf{E} \) 可表示为位移梯度张量的对称部分:
	\[
	\mathbf{E} = \begin{bmatrix}
		\varepsilon_{xx} & \frac{1}{2} \gamma_{xy} \\
		\frac{1}{2} \gamma_{xy} & \varepsilon_{yy}
		\end{bmatrix}
	\]
	其中应变分量分别为:
	\[
	\varepsilon_{xx} = \frac{\partial u}{\partial x}, \quad \varepsilon_{yy} = \frac{\partial v}{\partial y}, \quad \gamma_{xy} = \frac{\partial u}{\partial y} + \frac{\partial v}{\partial x}
	\]
	其中:
	\begin{itemize}
		\item \(\varepsilon_{xx}\) 和 \(\varepsilon_{yy}\) 分别是 \(x\) 和 \(y\) 方向的正应变,
		\item \(\gamma_{xy}\) 是工程剪应变,
		
		注意:工程剪应变 \( \gamma_{xy} \) 是总角度变化,
		而应变张量中的剪应变项为 \( \varepsilon_{xy}=\frac{1}{2} \gamma_{xy} \),以符合张量的对称性要求。
		\item \(\frac{\partial u}{\partial x}\)、\(\frac{\partial u}{\partial y}\)、
		\(\frac{\partial v}{\partial x}\)、\(\frac{\partial v}{\partial y}\) 是位移场的偏导数。
	\end{itemize}
\end{example}
\begin{defn}
	对数应变logrithmic strain(真应变true strain或亨奇应变Hencky strain)描述大变形的应变度量。

假设材料的初始长度为 \( L_0 \),变形后的长度为 \( L \)。在变形过程中,考虑每一微小增量的应变,应变增量可表示为:
\[
d\epsilon = \frac{dL}{L}
\]
对上述微分方程从初始长度 \( L_0 \) 到最终长度 \( L \) 进行积分,得到对数应变:
\[
\epsilon_{\text{true}} = \int_{L_0}^{L} \frac{dL}{L} = \ln\left(\frac{L}{L_0}\right)
=\ln(1+\varepsilon)
\]
因此,对数应变(亨奇应变)的定义为:
\[
\epsilon_{\text{true}} = \epsilon_{\text{Hencky}} = \ln\left(\frac{L}{L_0}\right)
\]
\end{defn}
\begin{defn}
	格林应变 (Green-Lagrange) 是用于描述大变形的应变度量,基于参考构形中的长度变化定义。
	
	设参考构形中的位置向量为 \(\vec{\X}\),变形后的位置向量为 \(\vec{x}\),变形梯度张量 \(\vvec{F}\) 定义为:
\[
\vvec{F} = \frac{\partial \vec{x}}{\partial \vec{X}}
\]
考虑参考构形中的线元 \(d\vec{\X}\) 和变形后的线元 \(d\vec{x}\),其关系为:
\[
d\vec{x} = \vvec{F} \cdot d\vec{\X}
\]
线元长度的平方变化为:
\[
|d\vec{x}|^2 - |d\vec{\X}|^2 = d\vec{\X} \cdot (\vvec{F^T} \vvec{F} - \vvec{I}) \cdot d\vec{\X}
\]
其中 \(\vvec{F^T}\) 是变形梯度张量的转置,\(\vvec{I}\) 是单位张量。定义格林应变张量 \(\vvec{E}\) 为:
\[
\vvec{E} = \frac{1}{2} (\vvec{F^T} \vvec{F} - \vvec{I})=
\frac{1}{2}(\vvec{C}-\vvec{I})=\frac{1}{2}\sum(\lambda_\Gamma^2-1)\vec{m_\Gamma}\otimes\vec{m_\Gamma}
\]
其中 \(\lambda_\Gamma\) 是主伸长比。因此,线元长度的平方变化可表示为:
\[
|d\vec{x}|^2 - |d\vec{\X}|^2 = 2 \, d\vec{\X} \cdot \vvec{E} \cdot d\vec{\X}
\]
\end{defn}
\begin{defn}
	Almansi strain是一种基于变形后构形(当前构形)的应变度量,适用于有限变形分析。

	设参考构形中的位置向量为 \(\vec{\X}\),变形后的位置向量为 \(\vec{x}\),变形梯度张量 \(\vvec{F}\) 定义为:
\[
\vvec{F} = \frac{\partial \vec{x}}{\partial \vec{X}}
\]
考虑参考构形中的线元 \(d\vec{\X}\) 和变形后的线元 \(d\vec{x}\),其关系为:
\[
d\vec{x} = \vvec{F} \cdot d\vec{\X}
\]
线元长度的平方变化为:
\[
|d\vec{x}|^2 - |d\vec{\X}|^2 = d\vec{x}^2-(\vvec{F}^{\lmone}d\vec{x})\cdot(\vvec{F}^{\lmone}d\vec{x})
=d\vec{x}\cdot(\vvec{I}-\vvec{F}^{\lmT}\vvec{F}^{\lmone})\cdot d\vec{x}
\]
其中 \(\vvec{I}\) 是单位张量。定义Almansi strain张量 \(\vvec{e}\) 为:
\[
	\vvec{e}=\frac{1}{2} (\vvec{I}-\vvec{F}^{\lmT}\vvec{F}^{\lmone})
	= \frac{1}{2} (\vvec{I}-\vvec{B}^{\lmone})
	= \frac{1}{2} (\vvec{I}-\vvec{c})
	=\frac{1}{2}\sum(\lambda_\gamma^2-1)\vec{n_\gamma}\otimes\vec{n_\gamma}
\]
因此,线元长度的平方变化可表示为:
\[
|d\vec{x}|^2 - |d\vec{\X}|^2 = 2 \, d\vec{x} \cdot \vvec{e} \cdot d\vec{x}
\]
\end{defn}
\section{class 33}
\begin{defn}
	Hill strain measures (希尔应变度量)

	Rodney Hill,1968---a general class 通类

	strain measure function(Lagrangian type)
	\[
	\vvec{E}_{Hill}=\vvec{f}(\vvec{U})=\vvec{f}(\sqrt{\vvec{C}})=\sum f(\lambda_\Gamma)
	\vec{m_\Gamma}\otimes\vec{m_\Gamma}
	\]

	\(f(\lambda_\Gamma)\) satisfies:
	\begin{itemize}
		\item $\lambda_\Gamma=1$ ,no deformation, $f(1)=0$
		\item $\frac{df}{d\lambda_\Gamma}>0$ ,strain is a increasing function
		\item \(\frac{df}{d\lambda_\Gamma}|_{\lambda_\Gamma=1}=
		f'(1)=1\Rightarrow df=d\lambda_\Gamma\)
	\end{itemize}
\begin{proof}
		\begin{gather*}
			\lambda_\Gamma\approx1+\varDelta\lambda,\varDelta\lambda\ll1\\
			small\; strain\quad\epsilon=\frac{1+\varDelta\lambda-1}{1}=\varDelta\lambda\\
			f(\lambda_\Gamma)\approx f(1+\varDelta\lambda)=f(1)+f'(1)\varDelta\lambda
			+o((\varDelta\lambda)^2)=f'(1)\varDelta\lambda\\
			only\; if\; f'(1)=1,\quad f(\lambda_\Gamma)=\varDelta\lambda\\
			df\approx f(1+d\lambda)-f(1)=d\lambda
		\end{gather*}
\end{proof}

	strain measure function(Lagrangian type)
	\[
		\vvec{e}_{Hill}=\vvec{f}(\vvec{V})=\vvec{f}(\sqrt{\vvec{B}})=\sum f(\lambda_\gamma)
		\vec{n_\gamma}\otimes\vec{n_\gamma}
	\]

	两种描述的关联
	\begin{align*}
		\vvec{E}_{Hill}&=\vvec{R^T}\vvec{e}_{Hill}\vvec{R}\quad\quad
		\vvec{R}=\sum\vec{n_\gamma}\otimes\vec{m_\Gamma}\\
		&=\sum(\vec{m_\Gamma}\otimes\vec{n_\gamma})\cdot f(\lambda_\gamma)
		(\vec{n_\gamma}\otimes\vec{n_\gamma})\cdot(\vec{n_\gamma}\otimes\vec{m_\Gamma})\\
		&=\sum f(\lambda_\gamma)\vec{m_\Gamma}\otimes\vec{m_\Gamma}
	\end{align*}
\end{defn}
\begin{defn}
	Seth 应变度量是一种广义的应变度量,适用于有限变形分析。
	
	\textbf{1. 拉格朗日法(基于参考构形)}
	Seth 应变度量定义为右 Cauchy-Green 变形张量 \(\vvec{C}\) 的函数:
	\[
	\vvec{E}^{(m)} = \frac{1}{2m} (\vvec{C}^m - \vvec{I}),
	\]
	其中:
	\begin{itemize}
		\item \(\vvec{C} = \vvec{F}^{\lsup{T}} \vvec{F}\) 是右 Cauchy-Green 变形张量;
		\item \(\vvec{F} = \frac{\partial \vec{x}}{\partial \vec{\X}}\) 是变形梯度张量;
		\item \(\vec{\X}\) 为参考构形位置向量,\(\vec{x}\) 为当前构形位置向量;
		\item \(m \in \mathbb{R}\) 为控制应变形式的参数;
		\item \(\vvec{I}\) 是单位张量。
	\end{itemize}

	\textbf{2. 欧拉法(基于当前构形)}
	Seth 应变度量定义为左 Cauchy-Green 变形张量 \(\vvec{B}\) 的逆的函数:
	\[
	\vvec{e}^{(m)} = \frac{1}{2m} (\vvec{I} - \vvec{B}^{-m}),
	\]
	其中:
	\begin{itemize}
    \item \(\vvec{B} = \vvec{F} \vvec{F}^{\lsup{T}}\) 是左 Cauchy-Green 变形张量;
    \item \(\vvec{F}^{-1} = \frac{\partial \vec{\X}}{\partial \vec{x}}\) 是变形梯度张量的逆;
    \item \(\vvec{B}^{-m}\) 表示 \(\vvec{B}\) 的逆的 \(m\) 次幂;
    \item 其他符号含义与拉格朗日法定义一致。
\end{itemize}

Seth 应变度量在不同 \(m\) 值下退化为常见的应变度量:
\begin{itemize}
    \item \textbf{当 \(m = 1\) 时}:
        \begin{itemize}
            \item 拉格朗日法退化为 Green-Lagrange 应变:\(\vvec{E}^{(1)} = \frac{1}{2}(\vvec{C} - \vvec{I})\);
            \item 欧拉法退化为 Almansi 应变:\(\vvec{e}^{(1)} = \frac{1}{2}(\vvec{I} - \vvec{B}^{-1})\)。
        \end{itemize}
    \item \textbf{当 \(m = 1/2\) 时}:
        \begin{itemize}
            \item 拉格朗日法退化为 Biot 应变:\(\vvec{E}^{(1/2)} = \vvec{U} - \vvec{I}\)(\(\vvec{C} = \vvec{U}^2\));
            \item 欧拉法退化为 Eulerian Biot 应变:\(\vvec{e}^{(1/2)} = \vvec{I} - \vvec{V}^{-1}\)(\(\vvec{B} = \vvec{V}^2\))。
        \end{itemize}
    \item \textbf{当 \(m \to 0\) 时}:
        \begin{itemize}
            \item 拉格朗日法退化为 Hencky(对数)应变:\(\vvec{E}^{(0)} = \frac{1}{2} \ln \vvec{C}\);
            \item 欧拉法退化为 Eulerian Hencky 应变:\(\vvec{e}^{(0)} = \frac{1}{2} \ln \vvec{B}\)。
        \end{itemize}
\end{itemize}

两种描述通过变形梯度张量 \(\vvec{F}\) 关联:
\[
\vvec{e}^{(m)} = \vvec{F}^{\lmT} \vvec{E}^{(m)} \vvec{F}^{-1}, \quad \vvec{E}^{(m)} = \vvec{F}^{\lsup{T}} \vvec{e}^{(m)} \vvec{F}.
\]
\end{defn}
\begin{add}
	line element, \(d\vec{x} = \vvec{F} \cdot d\vec{\X}=d\vec{\X} \cdot \vvec{F^T}\)	
	
	volume element, \(J=\frac{dv}{d\mathtt{V}}=\det\vvec{F}\)
\end{add}
\begin{defn}
	area element
	\begin{align*}
		dv&=d\vec{x}\cdot d\vec{a}=Jd\mathtt{V}=Jd\vec{\X}\cdot d\vec{\mathtt{A}}\\
		&=d\vec{\X}\vvec{F}^{\lsup{T}}\cdot d\vec{a}\\
		\Rightarrow&d\vec{\X}\left(\vvec{F}^{\lsup{T}}\cdot d\vec{a}-Jd\vec{\mathtt{A}}\right)=0\\
		\Rightarrow&d\vec{a}=J\vvec{F}^{\lmT}d\vec{\mathtt{A}}
		=\left(\text{Cof}\vvec{F}\right)d\vec{A}\quad\quad\text{Nanson's formula}
	\end{align*}
\end{defn}
\begin{add}
	Nanson's Formula \textbf{(此部分内容AIGC)}

	Nanson's formula 是连续介质力学中描述变形过程中面积元矢量变换的重要公式。其数学表达式为:
	\[
	d\vec{a} = J \vvec{F}^{\lmT} d\vec{A} = (\text{Cof}\, \vvec{F}) d\vec{A}
	\]
	其中:
	\begin{itemize}
		\item \(d\vec{a}\) 是变形后的面积元矢量。
		\item \(d\vec{A}\) 是变形前的面积元矢量。
		\item \(\vvec{F}\) 是变形梯度张量(\(\vvec{F} = \frac{\partial \vec{x}}{\partial \vec{X}}\),其中 \(\vec{x}\) 是变形后的坐标,\(\vec{X}\) 是变形前的坐标)。
		\item \(J = \det(\vvec{F})\) 是变形梯度张量的行列式,表示体积变化率。
		\item \(\text{Cof}\, \vvec{F}\) 是 \(\vvec{F}\) 的余因子矩阵(cofactor matrix),满足 \(\text{Cof}\, \vvec{F} = J \vvec{F}^{\lmT}\)。
	\end{itemize}
	\begin{center}
	\textbf{物理意义}
	\end{center}

	Nanson's formula 描述了变形过程中面积元矢量的变换关系,其物理意义可以从以下几个方面理解:
	
	\textbf{1. 面积元的缩放}
	\begin{itemize}
		\item 公式中的 \(J = \det(\vvec{F})\) 表示体积变化率。如果 \(J > 1\),表示材料在变形过程中体积膨胀;如果 \(J < 1\),表示体积收缩。
		\item 面积元的大小也会随 \(J\) 变化,具体表现为 \(|d\vec{a}| = J |\vvec{F}^{\lmT} d\vec{A}|\)。
	\end{itemize}
	
	\textbf{2. 法向量的旋转}
	\begin{itemize}
		\item 变形梯度张量的逆的转置 \(\vvec{F}^{\lmT}\) 描述了法向量的旋转。变形后的法向量 \(d\vec{a}\) 与变形前的法向量 \(d\vec{A}\) 通过 \(\vvec{F}^{\lmT}\) 相关联。
		\item 这种旋转反映了材料在变形过程中方向的变化。
	\end{itemize}
	
	\textbf{3. 面积元矢量的变换}
	\begin{itemize}
		\item 公式整体描述了变形前后面积元矢量的变换关系,结合了缩放和旋转效应。
		\item 特别地,\(\text{Cof}\, \vvec{F}\) 是变形梯度张量的余因子矩阵,它直接与面积元的变换相关。
	\end{itemize}
	\begin{center}
	\textbf{应用}
	\end{center}

	Nanson's formula 在连续介质力学中有重要应用,特别是在有限变形理论中。以下是其主要应用场景:
	
	\textbf{1. 应力变换}
	在连续介质力学中,应力的描述可以分为两种形式:
	\begin{itemize}
		\item \textbf{Cauchy 应力(真实应力)} \(\vvec{\sigma}\):
			\begin{itemize}
				\item Cauchy 应力定义在变形后的构形上,表示单位变形后面积上的力。
				\item 其数学表达式为:
					\[
					\vvec{\sigma} = \frac{d\vec{f}}{d\vec{a}}
					\]
					其中 \(d\vec{f}\) 是作用在变形后面积元 \(d\vec{a}\) 上的力。
			\end{itemize}
		\item \textbf{第一类 Piola-Kirchhoff 应力(名义应力)} \(\vvec{P}\):
			\begin{itemize}
				\item 第一类 Piola-Kirchhoff 应力定义在变形前的构形上,表示单位变形前面积上的力。
				\item 其数学表达式为:
					\[
					\vvec{P} = \frac{d\vec{f}}{d\vec{A}}
					\]
					其中 \(d\vec{f}\) 是作用在变形前面积元 \(d\vec{A}\) 上的力。
			\end{itemize}
		\item 两者之间的关系可以通过 Nanson's formula 推导得到:
			\begin{align*}
				\vvec{P} &= J \vvec{\sigma} \vvec{F}^{\lsup{-T}}=\vvec{\sigma} J\vvec{F}^{\lsup{-T}}
				=\vvec{\sigma}\text{Cof}\vvec{F}\\
				&=J\left(\sigma_{ij}\vec{e_i}\otimes\vec{e_j}\right)\cdot
				\left(\frac{\pa \X_K}{\pa x_l}\vec{e_l}\otimes\vec{e_K}\right)\\
				&=J\sigma_{ij}\frac{\pa \X_K}{\pa x_j}\vec{e_i}\otimes\vec{e_K}\quad\quad two\mathminus point\; tensor
			\end{align*}
						
	\end{itemize}
	
	\textbf{2. 有限元分析}
	\begin{itemize}
		\item 在有限元方法中,Nanson's formula 用于计算变形后的面积元,从而准确描述边界条件和载荷的变换。
	\end{itemize}
\end{add}
\section{class 34}
\begin{defn}
	第二类 Piola-Kirchhoff 应力 (拉格朗日型)
	\begin{align*}
		\vvec{T}&=\vvec{F}^{\lmone}\vvec{P}=\vvec{F}^{\lmone} J\vvec{\sigma}\vvec{F}^{\lmT}\\
		&=J\left(\frac{\pa \X_A}{\pa x_b}\vec{e_A}\otimes\vec{e_b}\right)\cdot\left(\sigma_{ij}\vec{e_i}\otimes\vec{e_j}\right)\cdot
		\left(\frac{\pa \X_K}{\pa x_l}\vec{e_l}\otimes\vec{e_K}\right)\\
		&=J\frac{\pa \X_A}{\pa x_b}\sigma_{ij}\frac{\pa \X_K}{\pa x_l}\vec{e_A}\otimes\vec{e_K}
	\end{align*}
	\[\vvec{T}^{\lsup{T}}=(\vvec{F}^{\lmone} J\vvec{\sigma}\vvec{F}^{\lmT})^T
	=\vvec{F}^{\lmone} J\vvec{\sigma}\vvec{F}^{\lmT}=\vvec{T}\]
\end{defn}
\begin{defn}
	Kirchhoff 应力 (欧拉型)
	\[\vvec{\tau}=J\vvec{\sigma}\]
\end{defn}
\begin{defn}
	功共轭 (Work Conjugate)
	\begin{align*}
		\dot{W}=\int \dot{w}\,dv&=\int \vvec{\sigma}:\vvec{d}\,dv
		=\int J\vvec{\sigma}:\vvec{d}\,d\mathtt{V}\\
		&=\int \vvec{\tau}:\vvec{d}\,d\mathtt{V}\quad\quad\vvec{d}\quad\text{strain rate}
	\end{align*}
	\begin{table}[ht]
		\centering
		\caption{功共轭关系对照表}
		\label{tab:work_conjugate}
		\begin{tabular}{lll}
		\toprule
		\textbf{力学变量} & \textbf{共轭变量} & \textbf{领域} \\
		\midrule
		柯西应力张量 $\vvec{\sigma}$ & 应变率张量 $\vvec{d}$ & 连续介质力学 \\
		Kirchhoff应力 $\vvec{\tau}$ & 应变率张量 $\vvec{d}$ & 连续介质力学 \\
		名义应力 $\vvec{P}$ & 变形梯度率 $\dot{\vvec{F}}$ & 非线性力学 \\
		第二类Piola-Kirchhoff应力 $\vvec{T}$ & Green-Lagrange应变率 $\dot{\vvec{E}}$ & 有限变形理论 \\
		力矢量 $\vec{F}$ & 位移矢量 $\vec{u}$ & 质点力学 \\
		热流矢量 $\vec{q}$ & 温度梯度 $-\nabla T$ & 热力学 \\
		偶应力张量 $\vvec{m}$ & 曲率张量 $\vvec{\kappa}$ & 微极连续介质力学 \\
		\bottomrule
		\end{tabular}
		\end{table}
		\begin{center}
			功共轭关系式
		\[\boxed{\dot{w}=J\vvec{\sigma}:\vvec{d}=\vvec{\tau}:\vvec{d}
		=\vvec{P}:\dot{\vvec{F}}=\vvec{T}:\dot{\vvec{E}}}\]
		\end{center}
	\begin{gather*}
		\dot{\omega}=\vvec{\tau}:\vvec{d}=\vvec{\tau}:(\vvec{l}-\vvec{\omega})\\
		\because\;\vvec{\tau}=\vvec{\tau}^{\lsup{T}},\vvec{\omega}^{\lsup{T}}=-\vvec{\omega}\quad
		\therefore\;\vvec{\tau}:\vvec{\omega}=0\\
		\dot{\omega}=\vvec{\tau}:\vvec{l}=\vvec{\tau}:\dot{\vvec{F}}\vvec{F}^{\lmone}\\
		\because\;\vvec{A}:(\vvec{B}\vvec{C})=(\vvec{B}^{\lsup{T}}\vvec{A}):\vvec{C}
		=(\vvec{A}\vvec{C}^{\lsup{T}}):\vvec{B}\\
		\dot{\omega}=\vvec{\tau}\vvec{F}^{\lmT}:\dot{\vvec{F}}=J\vvec{\sigma}\vvec{F}^{\lmT}:\dot{\vvec{F}}
		=\vvec{P}:\dot{\vvec{F}}
	\end{gather*}
	\begin{gather*}
		\dot{\omega}=J\vvec{F}^{\lmone}\vvec{F}\vvec{\sigma}\vvec{F}^{\lmT}:\dot{\vvec{F}}
		=\vvec{F}^{\lmone}J\vvec{\sigma}\vvec{F}^{\lmT}:(\vvec{F}^{\lsup{T}}\dot{\vvec{F}})
		=\vvec{T}:(\vvec{F}^{\lsup{T}}\dot{\vvec{F}})\\
		\because\;\vvec{T}=\vvec{T}^{\lsup{T}}\quad
		\therefore\;\dot{\omega}=\vvec{T}:\frac{\vvec{F}^{\lsup{T}}\dot{\vvec{F}}+(\vvec{F}^{\lsup{T}}\dot{\vvec{F}})^T}{2}
		=\vvec{T}:\frac{\vvec{F}^{\lsup{T}}\dot{\vvec{F}}
		+\dot{\vvec{F}^{\lsup{T}}}\vvec{F}}{2}=\vvec{T}:\dot{\vvec{E}}\\
		\vvec{E}=\frac{\vvec{C}-\vvec{I}}{2}=\frac{1}{2}(\vvec{F}^{\lsup{T}}\vvec{F}-\vvec{I})\\
		\dot{\vvec{E}}=\frac{1}{2}(\vvec{F}^{\lsup{T}}\dot{\vvec{F}}
		+\dot{\vvec{F}^{\lsup{T}}}\vvec{F})
	\end{gather*}
	\begin{zhu}
		计算双点积 $\vvec{\tau} : \vvec{\omega}$:
		\[
		\vvec{\tau} : \vvec{\omega} = \sum_{i,j} \tau_{ij} \omega_{ij}
		\]
		
		由于 $\vvec{\tau}$ 是对称的,$\tau_{ij} = \tau_{ji}$;而 $\vvec{\omega}$ 是反对称的,$\omega_{ij} = -\omega_{ji}$。因此,可以将双点积改写为:
		\[
		\vvec{\tau} : \vvec{\omega} = \sum_{i,j} \tau_{ij} \omega_{ij} = \sum_{i,j} \tau_{ji} (-\omega_{ji})
		\]
		
		将下标 $i$ 和 $j$ 互换(因为求和是对所有 $i,j$ 进行的):
		\[
		\vvec{\tau} : \vvec{\omega} = -\sum_{i,j} \tau_{ij} \omega_{ij} = -\vvec{\tau} : \vvec{\omega}
		\quad \Rightarrow \quad (\vvec{\tau} : \vvec{\omega}) = 0
		\]
	\end{zhu}
\end{defn}
\begin{defn}
	速度梯度 (velocity gradient)
		\[\vvec{l}=\vec{v}\nabla_{\vec{x}}=\frac{\pa \vec{v}}{\pa \vec{x}}
		=\frac{\pa}{\pa \vec{\X}}\left(\frac{d\vec{x}}{dt}\right)\vvec{F}^{\lmone}
		=\frac{d}{dt}\left(\frac{\pa \vec{x}}{\pa \vec{\X}}\right)\vvec{F}^{\lmone}
		=\dot{\vvec{F}}\vvec{F}^{\lmone}\]	
	
	应变率 (strain rate)
	\(\vvec{d}=\frac{\vvec{l}+\vvec{l}^{\lsup{T}}}{2}\; ,\; \vvec{d}^{\lsup{T}}=\vvec{d}\), 
	旋率 (spin)
	\(\vvec{\omega}=\frac{\vvec{l}-\vvec{l}^{\lsup{T}}}{2}\; ,\; \vvec{\omega}^{\lsup{T}}=-\vvec{\omega}\)
\end{defn}
\begin{example}
	\begin{gather*}
		\vvec{F}^{\lmone}\vvec{F}=\vvec{I}\quad\Rightarrow\quad
		\dot{\vvec{F}^{\lmone}}\vvec{F}+\vvec{F}^{\lmone}\dot{\vvec{F}}=\vvec{0}\\
		\because\; \vvec{l}=\dot{\vvec{F}}\vvec{F}^{\lmone}\quad
		\therefore\; \dot{\vvec{F}^{\lmone}}=-\vvec{F}^{\lmone}\vvec{l}
	\end{gather*}
\end{example}
\begin{example}
	\[
		\dot{\vvec{F}^{\lmT}}=\left(\dot{\vvec{F}^{\lmone}}\right)^T
		=(-\vvec{F}^{\lmone}\vvec{l})^T=-\vvec{l}^{\lsup{T}}\vvec{F}^{\lmT}
	\]
\end{example}
\begin{example}
	\begin{gather*}
		\frac{\pa J}{\pa \vvec{F}}=\frac{\pa \det\vvec{F}}{\pa \vvec{F}}
		=(\det\vvec{F})\vvec{F}^{\lmT}=J\vvec{F}^{\lmT}\\
		\dot{J}=\frac{\pa J}{\pa \vvec{F}}:\dot{\vvec{F}}=J tr(\vvec{F}^{\lmone}\dot{\vvec{F}})
		=J tr(\dot{\vvec{F}}\vvec{F}^{\lmone})=J tr\vvec{l}\\
		\because\; \vvec{l}=\vvec{d}+\vvec{\omega}\quad,\quad tr\vvec{\omega}=0\\
		\dot{J}=Jtr\vvec{d}
	\end{gather*}
\end{example}
\begin{example}
	\begin{gather*}
		d\vec{a}=J\vvec{F}^{\lmT}d\vec{A}\\
		\because\; \dot{d\vec{A}}=\vec{0}\\
		\dot{d\vec{a}}=\dot{J\vvec{F}^{\lmT}}d\vec{A}
		=(\dot{J}\vvec{F}^{\lmT}+J\dot{\vvec{F}^{\lmT}})d\vec{A}
		= J \vvec{F}^{\lmT} \left( \text{div} \vec{v} \vvec{I} - \vvec{l}^{\lsup{T}} \right) d\vec{A}\\
		\because\; J\vvec{F}^{\lmT} d\vec{A} = d\vec{a} \\
		\dot{d\vec{a}} = \left( (\text{div} \vec{v}) \vvec{I} - \vvec{l}^{\lsup{T}} \right) d\vec{A}
	\end{gather*}
\end{example}
\begin{example}
	\begin{gather*}
		\dot{dv}=\dot{Jd\mathtt{V}}=\dot{J}d\mathtt{V}=J(div\vec{v})d\mathtt{V}\\
		\because\; Jd\mathtt{V}=dv\quad
		\therefore\; \dot{dv}=(div\vec{v})dv
	\end{gather*}
\end{example}
\begin{example}
	\begin{gather*}
		\vvec{E} = \frac{1}{2} (\vvec{F}^{\lsup{T}} \vvec{F} - \vvec{I})\\
		\dot{\vvec{E}}=\frac{\dot{\vvec{F}^{\lsup{T}}} \vvec{F}+\vvec{F}^{\lsup{T}} \dot{\vvec{F}}}{2}
		=\frac{\vvec{F}^{\lsup{T}}\vvec{l}^{\lsup{T}} \vvec{F}+\vvec{F}^{\lsup{T}}\vvec{F}^{\lsup{T}} \vvec{F}}{2}
		=\vvec{F}^{\lsup{T}}\frac{\vvec{l}^{\lsup{T}} +\vvec{F}^{\lsup{T}}}{2}\vvec{F}
		=\vvec{F}^{\lsup{T}}\vvec{d}\vvec{F}
	\end{gather*}

	operation
	\begin{gather*}
		\vvec{F}^{\lsup{T}}\cdot(\text{当前构形})\cdot\vvec{F}\quad\text{pull\textminus back 拉回}
		\quad\dot{\vvec{E}}=\vvec{F}^{\lsup{T}}\vvec{d}\vvec{F}\\
		\vvec{F}^{\lsup{T}}\cdot(\text{参考构形})\cdot\vvec{F}\quad\text{push\textminus forward 推前}
		\quad\vvec{d}=\vvec{F}^{\lmT}\dot{\vvec{E}}\vvec{F}^{\lmone}
	\end{gather*}
\end{example}
\begin{example}
	介质不可压缩的seven equivalent conditions
\begin{center}
		\begin{tabular*}{0.7\textwidth}{@{\extracolsep{\fill}}ccc@{}}
			\(J=\det\vvec{F}=\frac{dv}{d\mathtt{V}}=1\) & \(\dot{J}=0\) &	\(div\vec{v}=\nabla\cdot\vec{v}=0\)\\
			\(tr\vvec{d}=0\) & \(tr\vvec{l}=0\) & \(dv=constant\)\\
			\(\vvec{F}^{\lmT}:\dot{\vvec{F}}=0\) & & \\
		\end{tabular*}
\end{center}
\end{example}









%  ↑↑↑↑↑↑↑↑↑↑↑↑↑↑↑↑↑↑↑↑↑↑↑↑↑↑↑↑ 正文部分
\ifx\allfiles\undefined
\end{document}
\fi
